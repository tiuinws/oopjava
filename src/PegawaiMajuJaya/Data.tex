% Created 2020-04-18 Sat 20:49
% Intended LaTeX compiler: pdflatex
\documentclass[11pt]{article}
\usepackage[utf8]{inputenc}
\usepackage[T1]{fontenc}
\usepackage{graphicx}
\usepackage{grffile}
\usepackage{longtable}
\usepackage{wrapfig}
\usepackage{rotating}
\usepackage[normalem]{ulem}
\usepackage{amsmath}
\usepackage{textcomp}
\usepackage{amssymb}
\usepackage{capt-of}
\usepackage{hyperref}
\date{\today}
\title{}
\hypersetup{
 pdfauthor={},
 pdftitle={},
 pdfkeywords={},
 pdfsubject={},
 pdfcreator={Emacs 26.3 (Org mode 9.1.9)}, 
 pdflang={English}}
\begin{document}

\tableofcontents

\section{Intro}
\label{sec:org3b58515}
CV. Maju Jaya memiliki 40 orang pegawai, dimana ke-40 pegawainya tersebut terbagi menjadi 2 status kepegawaian,
yaitu sebagian pegawai tetap dan sisanya adalah pegawai kontrak.
Secara umum perbedaan antara keduanya adalah pegawai tetap selain mendapatkan gaji juga mendapatkan tunjangan sebesar Rp500.000,- ,
sedangkan pegawai kontrak hanya mendapatkan gaji saja.
Dari kasus di atas, dapat digambarkan class-class pegawai sebagai berikut:
\begin{enumerate}
\item Class PegawaiKontrak dengan atribut:
\begin{itemize}
\item noPeg	: no pegawai kontrak (diinputkan)
\item nama	: nama pegawai (diinputkan)
\item masaKontrak: lama kontrak pegawai (diinputkan)
\item kehadiran	: jumlah hari masuk dalam 1 bulan (diinputkan)
\item uangMakan	: besarnya Rp.25.000,- dikali kehadiran (tidak diinputkan)
\item gajiPokok	: besarnya gaji pokok yang diterima tiap bulan (diinputkan)
\end{itemize}
\item Class PegawaiTetap dengan atribut:
\begin{itemize}
\item noPeg	: no pegawai tetap (diinputkan)
\item nama	: nama pegawai (diinputkan)
\item kehadiran	: jumlah hari masuk dalam 1 bulan (diinputkan)
\item tunjangan	: besarnya Rp.500.000,- (tidak diinputkan)
\item uangMakan	: besarnya Rp.25.000,- dikali kehadiran (tidak diinputkan)
\item gajiPokok	: besarnya gaji pokok yang diterima tiap bulan (diinputkan)
\end{itemize}
\end{enumerate}
Method-method yang harus ada:
\begin{enumerate}
\item hitungGaji()	: untuk menghitung gaji keseluruhan pegawai, dimana untuk pegawai kontrak = uang makan + gajiPokok, pegawai tetap = tunjangan + uang makan + gajiPokok

\item lihatData()	: untuk menampilkan data-data pegawai secara lengkap beserta total gaji yang diterima
\end{enumerate}
Dari kedua class di atas, analisis dan desainlah superclass yang tepat untuk kasus tersebut. Setelah itu buatlah class utama yaitu class PegawaiMajuJaya (yang menggunakan class-class di atas) yang memiliki menu sebagai berikut:
\begin{enumerate}
\item Input Data Pegawai
\item Lihat Data Pegawai
\end{enumerate}
Pilihan Anda [1/2] : ……..
Tentukan pula modifier yang tepat untuk semua class di atas (termasuk superclass-nya, mana yang final class dan mana yang abstract class, sedangkan interface sifatnya opsional (boleh digunakan/boleh tidak digunakan)


\section{Data-Data}
\label{sec:org03057fc}
\begin{itemize}
\item Atributes
\begin{enumerate}
\item Kelas Pegawai
\begin{itemize}
\item noPeg
\item nama
\item kehadiran
\item uangMakan
\item gajiPokok
\end{itemize}
\item Kelas PegawaiKontrak
\begin{itemize}
\item masaKontrak
\end{itemize}
\item Kelas PegawaiTetap
\begin{itemize}
\item tunjangan
\end{itemize}
\end{enumerate}
\end{itemize}
\end{document}
